\section{Differentialgleichung (DGL)}
\subsection{Klassifizierung der DGL}
Für die Vorlesung relevante DGL:
* lineare, homogene DGL 1. Ordnung mit  konstante Koeffizienten
* lineare, inhomogene DGL 1. Ordnung mit  konstante Koeffizienten
* nicht lineare, homogene DGL 1. Ordnung mit  konstante Koeffizienten
* lineare, homogene DGL 2. Ordnung mit  konstante Koeffizienten 
* lineare, inhomogene DGL 2. Ordnung mit  konstante Koeffizienten 
* In ausnahmenfälle linare DGL höherer Ordnung

Generell ist zu sagen das inhomogene DGL immer gelöst werden in dem wir zuerst die homogene DGL lösen (in dem wir g(x) = 0) setzen. Danach berechnen wir die inhomogenen Gleichung (den Ansatz wählen wir in dem wir die Form von g(x) berücksichtigen \ref{sec:Ansaetze}). Für die allgemeine Lösung rechnen wird die beiden Lösungen zusammen. 

Sind Anfangswerte gegeben setzten wir diese in die Gleichung oder das Gleichungssystem ein und reduzieren somit die Variablen in der Gleichung.

\subsection{Lineare DGL 1. Ordnung ($y' + f(x) \cdot y = g(x)$)}
\subsubsection{Variation der Konstanten, homogen}
%Nach http://de.wikipedia.org/wiki/Variation_der_Konstanten#Motivation
Sei die DGL in der Form $y' = f(x) \cdot y + g(x)$ , so sei:\[
F(x) = \int f(t) dt.
\]
Dann ist $\{y_{Hom}(x) = c_1e^{F(x)}| c_1 \in \R\}$ die Menge aller Lösungen der homogenen Differenentialgleichung ($y' + f(x) \cdot y = 0$). 

\subsubsection{Variation der Konstanten, inhomogener Teil}
\fbox{%
        \parbox{1\linewidth}{%
\textit{Dieser Teil muss nicht berechnet werden (Herleitung).}\\
Als Ansatz für die Lösung des inhomogenen Problems setze man $y_p(x) = u(x)e^{F(x)}$.
\\
Ansatzfunktion ist \[
y_p'(x) = u(x)f(x)e^{F(x)} + u'(x)e^{F(x)} = y(x)f(x) + u'(x)e^{F(x)}
\]
Also löst $y$ die inhomogene Differntialgleichung \[
y_p'(x) = y(x)f(x) + g(x)
\]
genau dann, wenn\[
u'(x) = y(x)f(x) + g(x)
\]
gilt.
        }%
}

Also folgt\[
u(x) = \int  g(t)e^{-F(t)} dt
\]
Somit ist die Lösungsmenge von $y_p$
\[
\{ y_p'(x) = e^{F(x)} (u(x) + c_2) | c_2 \in \R \}
\]
Die Lösungsmenge der generellen Lösung der allgemeinen Lösung ist somit $y$:
\begin{align*}
&\{y\}=\{y = y_{Hom} + y_p\}\\
& =\{ y(x) =  c_1e^{F(x)} +  e^{F(x)} (u(x) + c_2) | c_1, c_2 \in \R\} 
\end{align*}
Gibt es nun einen Ansatz, kann diese menge durch Einsetzen des Funktionswert und Gleichsetzen mit dem Resultat genau bestummen werde in dem die Konstanten $c$  aufgelöst werden.\\
\\
Konkret reicht es also aus. $F$ und $u$ zu berechnen. 

\subsubsection{Beispiel mit Variation der Konstanten}
%Nach http://de.wikipedia.org/wiki/Variation_der_Konstanten#Motivation
Gegeben: $y' + x^2 \cdot y = 2x^2$\\
Somit:  $y' = f(x) \cdot y + g(x)$ mit $g(x) := 2x^2$ und $f(x) :=  - x^2$\\
Es gilt somit:\[
F(x) = \int f(t) dt. =  \int - x^2 dt. = -\frac{x^3}{3}
\]
Dann ist $\{y_{Hom}(x) = c_1e^{-\frac{x^3}{3}}| c_1 \in \R\}$ die Menge aller Lösungen der homogenen Differnentialgleichung ($y' + x^2\cdot y = 0$). \\

Nun berechnen wir\[
u(x) = \int  g(x)e^{-F(x)} dx. =  2 \int  x^2e^{\frac{x^3}{3}} dx. = 2e^{\frac{x^3}{3}}
\]
Somit ist die Lösungsmenge von $y_p$ $\{ y_p'(x) = e^{-\frac{x^3}{3}} (2e^{\frac{x^3}{3}} + c_2) | c_2 \in \R \} = \{ y_p'(x) =2 + c_3) | c_3 \in \R \}$
Die Lösungsmenge der generellen Lösung der allgemeinen Lösung ist somit $y$\[
\{ y(x) =  c_1e^{-\frac{x^3}{3}} +  2  | c_1 \in \R \}
\]


\subsubsection{Separation der Variablen}
Wenn $g(x) = 0$ ist, dann ist die DGL homogen. Falls $g(x) \neq 0$, so handelt
es sich um eine inhomogene DGL. Wenn die DGL in der Form $y' = g(x)f(y)$ ist, kann sie 
direkt mit  "Separation der Variablen" berechnet werden in dem das alle von y abhängigen 
Variablen auf die einte und alle von x abhängige Variablen auf die andere Seite der Gleichung
gebracht werden. Siehe das nachfolgende Beispiel.
\subsubsection{Beispiel direkterer Lösungsweg}
Gegeben: $y' + x^2 \cdot y = 2x^2$.
Direkt lösen:
\begin{equation*}
\begin{array}{r l |l}
y' + x^2 \cdot y &= 2x^2\\
\frac{dy}{dx} + x^2 \cdot y &= 2x^2 \quad & -(x^2 \cdot y)\\
\frac{dy}{dx} &= 2x^2 -x^2 \cdot y \quad & \text{vereinfachen}\\
\frac{dy}{dx} &= x^2 ( 2 - y) \quad & \div (2 - y) \\
\frac{\frac{dy}{dx}}{2 - y} &= x^2  & \int \text{ with respect to } x \\
\int \frac{\frac{dy}{dx}}{2 - y}  \, dx &= \int x^2 \, dx & \text{links: } \int \frac{g'(x)}{g(x)} \, dx = \ln|g(x)|\\ 
- ln |2 - y| &= \frac{x^3}{3} + c_1 & \cdot (-1) \\
ln |2 - y| &= - \frac{x^3}{3} - c_1 & e \\
e^{ln |2 - y|} &= e^{-\frac{1}{3}x^3 - c_1 } \\
2 - y &= e^{-\frac{1}{3}x^3 - c_1 }& - 2 \\
- y &= e^{-\frac{1}{3}x^3 - c_1 } - 2 & \cdot (-1)\\
y &= - e^{-\frac{1}{3}x^3 - c_1 } + 2 & \text{replace const} \\
\underline{\underline{y}} &= \underline{\underline{c_2 \cdot e^{-\frac{1}{3}x^3} + 2}}\\
\end{array} 
\end{equation*}


\subsection{Lineare DGL höherer Ordnung}
Hier geht es um DGL der Form:
$a_n y^{(n)}+a_{(n-1)}y^{(n-1)}+\ldots+a_1 y'+a_0=g(x)$.

Wieder unterscheiden wir homogene DGL ($g(x) = 0$) und inhomogene DGL ($g(x) \neq 0$).

Als ersten Schritt lösen wir für homogene und inhomogene DGL die homogene Version der DGL:
$a_n y^{(n)}+a_{(n-1)}y^{(n-1)}+\ldots+a_1 y'+a_0 y = 0$. Dazu ersetzen wir $y^{(n)}$ durch
$\lambda^n \cdot e^{\lambda x}$. Beispielsweise wird aus $a_2 y'' + a_1 y' + a_0 y = 0$ wird
$a_2 \lambda^2 e^{\lambda x} + a_1 \lambda e^{\lambda x} + a_0 e^{\lambda x} = 0$.

Diese neue Gleichung ist das charakteristische Polynom. Von diesem berechnen wir
als erstes die Nullstellen $\lambda_i$. Wir beachten, dass wir auch komplexe Nullstellen
miteinbeziehen. Auch merken wir uns die Vielfachheit einer Nullstelle.

Hat man die Nullstellen $\lambda_i$, so lösen für einfache Nullstellen $e^{\lambda_i x}$
die homogene DGL. Ist die Nullstelle $\lambda_i$ $k$-fach, so lösen
$e^{\lambda_i x}, x e^{\lambda_i x}, x^2 e^{\lambda_i x}, \ldots, x^{k-1} e^{\lambda_i x}$
die homogene DGL. Wir erhalten also die allgemeine homogene Lösung in einer ähnlichen Form wie:
$y_h = C_n e^{\lambda_n x} + 
\underbrace{C_{n-1} e^{\lambda_{n-1} x} + C_{n-2} x e^{\lambda_{n-1} x}}_{\lambda_{n-1} \text{: 2-fache Nullstelle}} + \ldots + 
\underbrace{C_3 x^2 e^{\lambda_1 x} + C_2 x e^{\lambda_1 x} + C_1 e^{\lambda_1 x}}_{\lambda_1 \text{: 3-fache Nullstelle}}$.

Ist $\lambda_i$ eine komplexe Nullstelle, so ist $\lambda_i$ der Form $\lambda_i = a + i \cdot b$.
Zu jeder komplexen Nullstelle gibt es auch eine konjugierte Nullstelle: $\lambda_k = a - i \cdot b$.
Aus diesem Grund lösen für die komplexe Nullstelle ($\lambda_i$) $e^{x (a + ib)}$ und
$e^{x (a - ib)}$ das homogene DGL. \underline{Achtung:} Nachfolgend lohnt es sich oft die
Eulersche Identität zu verwenden: $e^{i \cdot x} = \cos(x) + i \sin(x)$.

Die allgemeine Lösung der homogenen DGL haben wir nun gefunden. Die Unbekannten $C_i$
können gefunden werden, wenn genügend Punkte gegeben sind, an denen der Funktionswert bekannt ist.

\subsubsection*{Lineare DGL höherer Ordnung, inhomogen}

Hat man ursprünglich eine inhomogene DGL vorliegen, so muss man für die allgemeine Lösung
noch die partikuläre Lösung des inhomogenen DGL berechnen. Dazu werden die Unbekannten
$C_i$ durch Funktionen $u_i(x)$ ersetzt. So wird aus
$y_h = C_2 x e^{\lambda_1 x} + C_1 e^{\lambda_1 x} \Rightarrow
y_p = u_2(x) x e^{\lambda_1 x} + u_1(x) e^{\lambda_1 x}$ (eine doppelte Nullstelle).

Jetzt geht es darum die Funktionen $u_i(x)$ zu bestimmen, um sie in die vorherige
$y_p$-Gleichung einsetzen zu können. Dazu stellen wir $i$ Gleichungen auf.
Also so viele, wie wir unbekannte Funktionen $u_i(x)$ haben:
\begin{align*}
u_2(x)' (x e^{\lambda_1 x}) + u_1(x)' (e^{\lambda_1 x}) &= 0\\
u_2(x)' (x e^{\lambda_1 x})' + u_1(x)' (e^{\lambda_1 x})' &= g(x)
\end{align*}

Das Prinzip ist folgendes: Bis auf die letzte Gleichung, wird gleich $0$ gesetzt.
Die letzte Gleichung wird gleich $g(x)$ gesetzt.
Unsere unbekannten Funktionen werden jeweils einmal abgeleitet, egal in welcher
Gleichung wir sind. Pro Zeile, die man weiter runter geht, wird der Term mit $e^{\lambda_i x}$
jeweils einmal mehr abgeleitet. In der ersten Zeile wird zum Beispiel $e^{\lambda_1 x}$ nicht abgeleitet,
in der nächsten Gleichung wird es einmal abgeleitet. Hätten wir mehr Unbekannte Funktionen,
so würde in der folgenden Zeile zwei mal abgeleitet werden. Im Allgemeinen gilt also:
{\footnotesize
\begin{align*}
u_1(x)' y_{h1}(x) + u_2(x)' y_{h2}(x) + \ldots + u_n(x)' y_{hn}(x) &= 0\\
u_1(x)' y_{h1}(x)' + u_2(x)' y_{h2}(x)' + \ldots + u_n(x)' y_{hn}(x)' &= 0\\
u_1(x)' y_{h1}(x)'' + u_2(x)' y_{h2}(x)'' + \ldots + u_n(x)' y_{hn}(x)'' &= 0\\
&\ldots\\
u_1(x)' y_{h1}(x)^{(n-1)} + u_2(x)' y_{h2}(x)^{(n-1)} + \ldots + u_n(x)' y_{hn}(x)^{(n-1)} &= g(x)
\end{align*}
}

Diese Gleichungen werden nun jeweils aufgelöst, bis man $u_i(x)$ erhält. Um zu
$u_i(x)$ zu gelangen, muss auf dem Weg einmal die Gleichung auf beiden Seiten
integriert werden. Hat man alle $u_i(x)$, so setzt man diese in unsere
ursprüngliche $y_p$ Gleichung ein.

Nun kann die allgemeine Lösung des inhomogenen DGL berechnet werden. Dazu
summiert man $y_h$ und $y_p$: $y = y_h + y_p$. Dies ist die allgemeine Lösung.
Hat man konkrete Punkte, an denen die Funktion ausgewertet wurde, so kann man
die Unbekannten $C_i$ berechnen.

\subsubsection{Euler}

$$ \underbrace{C_1 \cdot e^{(a+bi)x} + C_2 \cdot e^{(a-bi)x}}_{D_1 \cdot e^{ax}\sin(bx) + D_2 \cdot e^{ax}\cos(bx)}$$


\subsubsection{Beispiel}
Es soll $y'' + y = \frac{2}{\cos(x)}$ ausgerechnet werden.

Zuerst sehen wir uns das homogene DGL an:
\begin{align*}
y'' + y &= 0\\
\Rightarrow \lambda^2 e^{\lambda x} + e^{\lambda x} &= 0\\
\Leftrightarrow e^{\lambda x} (\lambda^2 + 1) &=0 \\
\Rightarrow \lambda^2 + 1 &= 0\\
\Leftrightarrow \lambda^2 &= -1 \quad
\Rightarrow \underline{\lambda_1 = i},\, \underline{\lambda_2 = -i}
\end{align*}

Somit ist die allgemeine Lösung des homogenen DGL:
\begin{align*}
y_h &= C_1 \cdot e^{ix} + C_2 \cdot e^{-ix}\\
&= C_1 (\cos(x) + i\sin(x)) + C_2(\cos(x) - i\sin(x))\\
&= \sin(x) \underbrace{(i C_1 + i C_2)}_{ = D_1} + \cos(x) \underbrace{(C_1 + C_2)}_{= D_2}\\
&= \underline{D_1 \sin(x) + D_2 \cos(x) = y_h}
\end{align*}

Da es sich um ein inhomogenes DGL handelt, berechnen wir als nächstes die partikuläre
Lösung:
\begin{align*}
y_p &= \underbrace{u_1(x)}_{D_1 \text{ in } y_h} \sin(x) + \underbrace{u_2(x)}_{D_2 \text{ in } y_h} \cos(x)\\
&\Rightarrow \left|
	\begin{aligned}
		u_1(x)' \sin(x) + u_2(x)' \cos(x) &= 0\\
		u_1(x)' \sin(x)' + u_2(x)' \cos(x)' &= \frac{2}{\cos(x)}
	\end{aligned}
\right|\\
&= \ldots\\
&\Rightarrow u_1(x)' = -2 \tan(x),\, u_2(x)' = 2\\
&\Leftrightarrow u_1(x) = 2 \ln(\cos(x)),\, u_2(x) = 2x\\
&\Rightarrow \underline{y_p = 2 \ln(\cos(x)) \sin(x) + 2x \cos(x)}
\end{align*}

Da wir nun auch die partikuläre Lösung haben, können wir die allgemeine Lösung
des inhomogenen DGL berechnen:
$\underline{\underline{y}} = y_h + y_p = \underline{\underline{D_1 \sin(x) + D_2 \cos(x) + 2 \ln(\cos(x)) \sin(x) + 2x \cos(x)}}$

\subsection{Ansätze für partikuläre Lösung} \label{sec:Ansaetze}
	\textbf{Hinweis}: nur brauchbar für lineare DGL mit konstanten Koeffizienten \\
	\\
	Bezeichnungen:
	\begin{align*}
	P(x)  &  \quad \text{charakt. Polynom der DGL}  \\
	S_k(x) & \quad \text{polynomielle Störfunktion, Grad} \; k \\
	A, B & \quad \text{unbekannte Konstanten} \\
	R_k(x) = a_k x^k + . + a_1 x + a_0 & \quad \text{mit unbekannten Koeffizienten}
	\end{align*}
	
	
	
	\begin{tabular}{l|l}
		$g(x)$ & $Ansatz$ \\ \hline \hline
		$ c e^{mx} $  & $A e^{mx}$ , falls $P(m) \neq 0$\\
		&  $A x^q e^{mx}$, falls $m$ $q$-fache NST von $P$ \\ \hline
		$ S_k(x) $  & $R_k(x)$ , falls $P(0) \neq 0$\\
		&  $x^q R_k(x)$, falls $0$ $q$-fache NST von $P$ \\ \hline
		$ P_k(x) e^{mx} $  & $R_k(x) e^{mx}$ , falls $P(m) \neq 0$\\
		&  $x^q R_k(x) e^{mx}$, falls $m$ $q$-fache NST von $P$ \\ \hline
		$ \sin wx, \cos wx $  & $A \cos wx + B \sin wx$ , falls $P(\pm iw) \neq 0$\\
		&  $x^q (A \cos wx + B \sin wx)$, falls $\pm iw$ \\
		& $q$-fache NST von $P$ \\ \hline
		$ \sinh wx, \cosh wx $  & $A \cosh wx + B \sinh wx$ , falls $P(w) \neq 0$\\
		&  $x^q (A \cosh wx + B \sinh wx)$, falls $w$ \\
		& $q$-fache NST von $P$ \\
	\end{tabular}
