\section{Vektorräume}

%	\subsection{Unterraum}
	
	\begin{fdef}[Unterraum (auch Untervektorraum)]
		Folgende 2 bedinungen müssen für den Unterraum ($U \subseteq V$) gelten:
		\[
			a + b \in U, \quad \alpha a \in U \qquad 
			(\forall a,b \in U, \ \forall \alpha \in \E)
		\]
	\end{fdef}

	\textbf{Rechenbeispiel}
	
	Man nimmt $a, b \in U$, $\alpha \in \E$ und nimmt an $c,d~\in~U$. Nun zeige, 
	dass $c = a + b$ und $d = \alpha a$ und somit $c,~d~\in~U$.
	
	\begin{fdef}[aufgespanter Unterraum]
		Der von $a_1, \dots ,a_l \in V$ aufgespannte
		Unterraum wird bezeichnet mit:
		
		\[
			\spn{a_1,\dots,a_l} :\equiv
			\left\{
				\sum_{k=1}^{l} \gamma_k a_k \ ; \ \gamma_1, \dots ,\gamma_l \in \E
			\right\}
		\]
	\end{fdef}

	
	\begin{fdef}[Erzeugendensystem]
		Spannt einen Vektorraum $\spn{a_1, \dots a_l}$ auf.
	\end{fdef}
	
	
