\section{Taylorreihe \& -polynom}

\subsection{Taylorreihe}
Funktion $f$ wird in einer Umbegung eines bestimmten Punktes $a$ druch eine \textit{Potenzreihe} dargestellt.
\[
	P_{f,a}(x) = \sum_{k=0}^\infty \frac{f^{(k)}(a)}{k!} (x-a)^k 
	\ [= f(x)] \text{ (für x mit } |x-a| < r \text{)}
\]

\subsubsection{Beispiele}
$e^x = \sum_{k=0}^{\infty} \frac{x^k}{k!}$, $sin(x)$, $cos(x)$

\subsection{Taylorplynom (Tailorformel)}
\textit{Näherung} der Funktion im Punkt $a$ durch ein Polynom $n$-ter Ordnung:
$T_{f,a,n}(x) = T_n(x) = \sum_{k=0}^n \frac{f^{(k)}(a)}{k!} (x-a)^k$\newline
{\small
	$= f(a) + \frac{f'(a)}{1!}(x-a) + \frac{f''(a)}{2!}(x-a)^2 +
	\frac{f^{(3)}(a)}{3!}(x-a)^3 + \frac{f^{(4)}(a)}{4!}(x-a)^4\ldots$
}

\subsection{Satz zu den Tailorpolynomen}

\[
	f(x) = T_n(x) + R_n \text{ mit } R_n = \frac{f^{(n+1)}(\xi)}{(n+1)!} (x-a)^{n+1} \
	(\xi \in [a,x])
	% komplettere Formel:
	% R_{n,a} = \frac{1}{n!} \int_a^x f^{(n+1)}(t) dt
	% |R_{n,a}(x)| \leq \frac{1}{(n+1)!} max_{\xi \text{ zw. } a \& x} |f^{(n+1)}(\xi)|
	% |x-a|^{n+1}
\]

\subsection{Bemerkungen / Eigenschaften / Konvergenz}
\begin{itemize}
	\item Der Konvergenzradius kann 0 sein
	\item Falls Taylor-Reihe konvergiert, dann ist sie nicht notwendig gleich
	der Funktion, die sie beschreibt. Gegenbeispiel:
	$f(x) = \begin{cases}
	e^{-\frac{1}{x}} & x > 0\\
	0 & x \leq 0\end{cases}$
	\item Ist $f$ eine Potenzreihe, dann ist diese Potenzreihe auch die Taylor-Reihe
\end{itemize}