\section{Reihen}

\subsection{Definitionen}
Eine Reihe $\sum_{n = 1}^\infty a_n$ ist \underline{konvergent} mit Grenzwert
$s$, wenn die Folge der \underline{Partialsummen} $(S_m)$, $S_m :=
\sum_{n=1}^m a_n$ gegen $s$ konvergiert. Also wenn gilt: $S_m \to s$.

\begin{definition}[$\epsilon$-Kriterium]
	$\forall \epsilon > 0 \; \exists n_0 \in \N \; \forall m \geq n_0: \left|
	\sum_{n=1}^m a_n - s \right| < \epsilon$
\end{definition}

\begin{definition}[Absolute Konvergenz]\index{Konvergenz}
Wenn auch die Reihe der Absolutbeträge $\sum_{n=1}^\infty |a_n|$ konvergiert, so
heisst die Reihe absolut konvergent. Aus der absoluten Konvergenz folgt
Konvergenz. Der Umkehrschluss ist nicht möglich.
\end{definition}

\subsection{Rechenregeln Reihen}
Für \underline{konvergente} Reihen gilt:
\[
	\sum_{n=1}^\infty a_n = A, \sum_{n=1}^\infty b_n = B \Rightarrow
	\sum_{n=1}^\infty (\alpha a_n + \beta b_n) = \alpha A + \beta B
\]

\subsection{Konvergenzkriterien}

\subsubsection{Reihen Kriterien}
Achtung. Die nachfolgenden Kriterien sagen nur aus, ob die Reihen konvergiert
oder nicht. Sie sagen \underline{nicht} aus, gegen was sie konvergieren!

\paragraph{Cauchy-Kriterium}

Konvergiert $\sum_{n=1}^\infty a_n$, so ist $\lim_{n \to \infty} a_n = 0$.\\
Wenn also $\lim_{n \to \infty} a_n \neq 0$, so konvergiert die Reihe
\underline{nicht}.

\paragraph{Quotientenkriterium}
\[
\left| \frac{a_{n+1}}{a_n} \right| \to q. \quad \text{Dann gilt} \begin{cases}
q < 1 & \Rightarrow \sum_{n=1}^\infty a_n \text{ konvergiert absolut} \\
q = 1 & \Rightarrow \text{keine Aussage}\\
q > 1 & \Rightarrow \sum_{n=1}^\infty a_n \text{ divergiert}
\end{cases}
\]

\paragraph{Wurzelkriterium}
\[
\sqrt[n]{\left | a_n \right |} \to q. \quad \text{Dann gilt} \begin{cases}
q < 1 & \Rightarrow \sum_{n=1}^\infty a_n \text{ konvergiert absolut}\\
q = 1 & \Rightarrow \text{keine Aussage}\\
q > 1 & \Rightarrow \sum_{n=1}^\infty a_n \text{ divergiert}
\end{cases}
\]

\paragraph{Majorantenkriterium}
Ist $|a_n| \leq b_n$ und $\sum_{n=1}^\infty b_n$ konvergent, so konvergiert
$\sum_{n=1}^\infty a_n$ absolut.

\paragraph{Minorantenkriterium}
Ist $a_n \geq b_n \geq 0$ und $\sum_{n=1}^\infty b_n$ divergent, so divergiert
$\sum_{n=1}^\infty a_n$

\paragraph{Leibnizkriterium}
Wenn gilt:
\begin{itemize}
  \item $(a_n)$ ist alternierende Folge, d.h die Vorzeichen wechseln jedes Mal
  \item $a_n \to 0$ oder $|a_n| \to 0$
  \item $(|a_n|)$ ist monoton fallend
\end{itemize}
\ldots dann konvergiert $\sum_{n=1}^\infty a_n$

\subsection{Potenzreihe}
Die Potenzreihe hat die allgemeine Form
\[
\sum_{n=0}^\infty a_n (x - x_0)^n
\]

$x_0$ ist der Entwicklungspunkt der Potenzreihe und $(a_n)_{n \in \N}$ eine
beliebige Folge.

\subsubsection{Konvergenzradius}
Die Berechnung des Konvergenzradius ist für solche Reihen einfacher, da der
Faktor $(x - x_0)$ nicht analysiert werden muss. Entsprechend gilt für den
Konvergenzradius $r$:
$r = \frac{1}{\limsup_{n\to\infty} \sqrt[n]{\|a_n\|}}$ (Wurzelkriterium) bzw.
$r = \lim_{n\to\infty} \left | \frac{a_n}{a_{n+1}} \right |$
(Quotientenkriterium).

\subsubsection{Bsp: Von Bruch Reihe finden}
\begin{gather*}
	\frac{1}{x-5} = \frac{1}{5} \cdot \frac{1}{(\frac{x}{5})-1} 
	= - \frac{1}{5} \cdot \frac{1}{1-(\frac{x}{5})}
	= - \frac{1}{5} \cdot \sum_{k=0}^{\infty}(\frac{x}{5})^k \\
	= \sum_{k=0}^\infty - \frac{1}{5} (\frac{x}{5})^k
	\quad \text{mit } |\frac{x}{5}| < 1 \Rightarrow |x| < 5
\end{gather*}
