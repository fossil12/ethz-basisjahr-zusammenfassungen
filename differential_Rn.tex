\section{Differentialrechnung in $\R^n$}
Hier geht es um Funktionen $f: \R^n \to \R^m$, wobei $m=1$ gelten kann
($f: \R^n \to \R$). Solche Funktionen haben die allgemeine Form:
$f(x) = f(x_1, x_2, x_3, \ldots, x_n) = \begin{pmatrix}
f_1(x_1, x_2, x_3, \ldots, x_n)\\
f_2(x_1, x_2, x_3, \ldots, x_n)\\
\ldots\\
f_m(x_1, x_2, x_3, \ldots, x_n)
\end{pmatrix}$

Für nahezu alle Eigenschaften gilt: Die Vektorfunktion $f: \R^n \to \R^m$ hat
eine bestimmte Eigenschaft, wenn jede einzelne ihrer Komponenten
($f_1, f_2, \ldots, f_m$) die besagte Eigenschaft besitzen. Das Problem liegt
neu also nicht im Wertebereich, sondern vor allem in Definitionsbereich.

\subsection{Norm}
Eine Norm auf $\R^n$ ist die Funktion $\|\cdot\|: \R^n \to \R$ mit den folgenden
Eigenschaften:
\begin{itemize}
	\item $\forall x \in \R^n: \|x\| \geq 0$
	\item $\forall x \in \R^n: \|x\| = 0 \Leftrightarrow x = \vec{0}$
	\item $\forall x \in \R^n, \alpha \in \R: \|\alpha x\| = |\alpha| \|x\|$
	\item $\forall x,y \in \R^n: \|x + y\| \leq \|x\|+\|y\|$
\end{itemize}

\subsection{Partielle Differenzierbarkeit}
$f: \R^n \to \R^m$ ist in $a = (a_1, \ldots, a_n)$ partiell differenzierbar nach
der $i$-ten Variable $x_i$, wenn die Funktion
$f: x_i \to f(x_1, \ldots, x_i, \ldots, x_n)$ differnzierbar ist. Man berechnet
die partielle Ableitung also folgendermassen: Eine Funktion $f$ wird nach einer
Variable partiell differenziert, indem man alle anderen Variablen als Konstanten
behandelt und die Rechenregeln für Funktionen mit einer Variable anwendet.

\begin{satz}[Satz von Schwarz]
Ist $f$ nach $x$ und $y$ zweimal partiell differenzierbar und sind die gemischten
partiellen Ableitungen $f_{xy}$ und $f_{yx}$ stetig, so gilt: $f_{xy} = f_{yx}$.
\end{satz}

\subsection{Hesse-Matrix}
%Meiste Formeln kopiert von http://de.wikipedia.org/wiki/Hesse-Matrix
Die Hesse-Matrix ist eine Matrix, die in der mehrdimensionalen reellen Analysis ein Analogon zur zweiten Ableitung einer Funktion ist. Ist die Funktion zweimal stetig differenzierbar kann die Hesse-Matrix gebildet werden. Eine Hesse-Matrix ist symetrisch.

Für eine Funktion $f(x,y)$ sieht die Hesse Matrix wie folgt aus:

$$\operatorname{H}_f=
\begin{pmatrix}
\frac{\partial^2 f}{\partial^2 x}&\frac{\partial^2 f}{\partial x\partial y}\\
\frac{\partial^2 f}{\partial y\partial x}&\frac{\partial^2 f}{\partial^2 y}
\end{pmatrix}.$$

Verallgemeinert für die Funktion $f \colon D \subset \R^n \to \R$  (die ebenfalls zweimal stetig differenzierbar sein muss), sieht die Matrix wie folgt aus:

\begin{equation}\begin{split}\operatorname{H}(f)=\operatorname{H}_f=
\left(\frac{\partial^2f}{\partial x_i\partial x_j}\right)_{i,j=1,\dots, n}= \\
\begin{pmatrix}
\frac{\partial^2 f}{\partial x_1\partial x_1}&\frac{\partial^2 f}{\partial x_1\partial x_2}&\cdots&\frac{\partial^2  f}{\partial x_1\partial x_n}\\[0.5em]
\frac{\partial^2 f}{\partial x_2\partial x_1}&\frac{\partial^2 f}{\partial x_2\partial x_2}&\cdots&\frac{\partial^2  f}{\partial x_2\partial x_n}\\
\vdots&\vdots&\ddots&\vdots\\
\frac{\partial^2 f}{\partial x_n\partial x_1}&\frac{\partial^2 f}{\partial x_n\partial x_2}&\cdots&\frac{\partial^2  f}{\partial x_n\partial x_n}
\end{pmatrix}.\end{split}\end{equation}

Anmerkung: Die Determinate einer Matrix ist: 
$$\det A=\det
  \begin{pmatrix}
    a_{11} & a_{12} \\
    a_{21} & a_{22} 
  \end{pmatrix} 
= a_{11} a_{22} - a_{12} a_{21}.$$

\begin{equation}\begin{split}
\det A = \det
  \begin{pmatrix}
    a_{11} & a_{12} & a_{13} \\
    a_{21} & a_{22} & a_{23} \\
    a_{31} & a_{32} & a_{33} 
  \end{pmatrix}
\\= a_{11} a_{22} a_{33} +a_{12} a_{23} a_{31} + a_{13} a_{21} a_{32} 
\\ - a_{13} a_{22} a_{31} - a_{12} a_{21} a_{33} - a_{11} a_{23} a_{32}.
\end{split}\end{equation}

\subsection{Kritische Punkte}
Berechnen von lokalen Minima und Maxima. (Achtung: Rand seperat betrachten!)

\resizebox{!}{15cm}{%
%tikz graph inspiered by http://slopjong.de/2010/04/10/urinal-manual/
%content inspired by jörn http://www.youtube.com/watch?v=fdDuptml-2E
\tikzstyle{decision} = [diamond, draw, fill=blue!20, 
text width=4.5em, text badly centered, node distance=3cm, inner sep=0pt]

\tikzstyle{block} = [rectangle, draw, fill=blue!20, 
text width=4cm, text centered, rounded corners, minimum height=4em, minimum width=4.5cm]

\tikzstyle{line} = [draw, -latex']

%from http://tex.stackexchange.com/questions/60585/how-to-highlight-a-single-element-in-a-matrix
\newcommand\hlight[1]{\tikz[overlay, remember picture,baseline=-\the\dimexpr\fontdimen22\textfont2\relax]\node[rectangle,fill=yellow!50,rounded corners,fill opacity = 0.2,draw,thick,text opacity =1] {$#1$};} 


\begin{tikzpicture}[node distance = 3cm, auto]
% The nodes  
\node[block] (grad) {Grad(f) = 0 ergibt kritische Punkte ($\nabla f = \vec{0}$)};    
\node[block, below of=grad] (matrix) {Hesse-Matrix $\operatorname{H}_f$ berechnen};    
\node[block, below of=matrix] (det) {Determinante Berechnen $a_{11} a_{22} - a_{12} a_{21}$};    
\node [decision, below of=det, text width=2cm] (detS) {$det < 0$?};
\node [decision, below of=detS, text width=2cm] (detB) {$det > 0$?};
\node [decision, below of=detB, text width=2cm] (definit) {$a_{11} > 0$?};

\node[block, right of=detS, node distance=5cm] (sattel) {Sattelpunkt};
\node[block, right of=detB, node distance=5cm] (unb) {Unbestimmt};
\node[block, right of=definit, node distance=5cm] (max) {Lokales Maximum};
\node[block, below of=definit] (min) {Lokales Minimum};

\node[rectangle, right of=matrix , node distance=5cm] (hf) {$\operatorname{H}_f=
\begin{pmatrix}
\hlight{\frac{\partial^2 f}{\partial^2 x}}&\frac{\partial^2 f}{\partial x\partial y}\\
\\
\frac{\partial^2 f}{\partial y\partial x}&\frac{\partial^2 f}{\partial^2 y}
\end{pmatrix}$};


% The connections
\path [line] (grad) -- node [near start] {$(x_0,y_0)$} (matrix);

\path [line] (matrix) -- node [near start] {$\operatorname{H}_f$} (det);
\path [line] (det) -- node [near start] {$det$} (detS);
\path [line] (detS) -- node [near start] {Nein} (sattel);
\path [line] (detS) -- node [near start] {Ja} (detB);

\path [line] (detB) -- node [near start] {Nein} (unb);
\path [line] (detB) -- node [near start] {Ja} (definit);

\path [line] (definit) -- node [near start] {Ja} (max);
\path [line] (definit) -- node [near start] {Nein} (min);
\end{tikzpicture}
}
